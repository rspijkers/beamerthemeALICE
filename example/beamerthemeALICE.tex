% Copyright 2020 by Junwei Wang <i.junwei.wang@gmail.com>
%
% This file may be distributed and/or modified under the
% conditions of the LaTeX Project Public License, either version 1.3c
% of this license or (at your option) any later version.
% The latest version of this license is in
%   http://www.latex-project.org/lppl.txt

\documentclass[compress]{beamer}
% compress = slide titles on one line
% aspectratio is set to 16:9, default is "square" slides

\usepackage[english]{babel}
\usepackage{metalogo}
\usepackage{listings}
\usepackage{fontspec}
\usepackage{tikz}

\usepackage{subcaption}

\usetheme{ALICE}

%%% SHOW TOC AT START OF (SUB)SECTION %%%
% \AtBeginSection[]
% {
%   \begin{frame}[c,noframenumbering,plain]
%     \tableofcontents[sectionstyle=show/hide,subsectionstyle=show/show/hide]
%   \end{frame}
% }

\title{Presentation Title}
\subtitle{This is a subtitle (optional). You can even make it two lines long! Not three though...}
\author{Rik Spijkers}
\institute{Institute, optional}
\date{\today}

\begin{document}

%%% TITLE SLIDE %%%
\begin{frame}[plain,noframenumbering]
  \maketitle
\end{frame}

%%% TABLE OF CONTENTS %%%
\begin{frame}[c,plain,noframenumbering]
  \tableofcontents[]
\end{frame}

%%% OUTLINE %%%
\section{Section 1}
\subsection{Subsection 1}
\section{Section 2}
\subsection{Subsection 1}
\subsection{Subsection 2}

\begin{frame}[fragile]
  \frametitle{How to use this package}
  Simply include the following code in your preamble:

  \begin{lstlisting}[basicstyle = \ttfamily\small]
    \usetheme{ALICE}
  \end{lstlisting}

  See also the github page for more documentation: \\
  https://github.com/rspijkers/beamerthemeALICE

\end{frame}

\begin{frame}{The color scheme}
  \begin{description}
    \item \textcolor{AliceDarkBlue}{AliceDarkBlue} \quad \textcolor{AliceBlue}{AliceBlue} \quad \textcolor{AliceLightBlue}{AliceLightBlue} \quad \textcolor{AliceLighterBlue}{AliceLighterBlue}\\
    \item \textcolor{AliceRed}{AliceRed}\\
    \item \textcolor{AliceWhite}{AliceWhite}\\
    \item \textcolor{AliceGray}{AliceGray}\\
    \item 
    \item The background of this slide is in AliceBlack
  \end{description}
\end{frame}

\begin{frame}
  \frametitle{Blocks}
  \begin{block}{This is a Block}
    \[
      a^2 + b^2 = c^2
    \]
  \end{block}
  \centering
  \begin{minipage}{1.0\linewidth}
    \begin{block}{Horizontally-Aligned Block}
      \[
        \log xy = \log x + \log y
      \]
    \end{block}
  \end{minipage}
\end{frame}

\begin{frame}{Items}
  Itemize
  \begin{itemize}
    \item item 1
    \item item 2
  \end{itemize}

  \bigskip

  Enumerate
  \begin{enumerate}
    \item item 1
    \item item 2    
  \end{enumerate}
\end{frame}

\begin{frame}{Figures}
  \begin{figure}
    \centering
    \begin{tikzpicture}
      \draw [help lines,AliceRed,very thick] (0,0) grid (5,4);
    \end{tikzpicture}
    \caption{Credits to Ti\textit{k}Z}
  \end{figure}
\end{frame}

%% BACKUP %%% 
\section*{} % empty section to ensure that the backup is not counted as part of the last section

\begin{frame}{Backup slides: Also, this title is very long to show that line breaks in frame titles are handled correctly}
  This is a backup slide, it should not be in the TOC
\end{frame}

\end{document}

